% Created 2013-10-04 Fri 18:48
\documentclass[bigger]{beamer}
\usepackage[utf8]{inputenc}
\usepackage{hyperref}
\usepackage{graphicx}
\usepackage{longtable}
\usepackage{float}
\mode<beamer>{\usetheme{Madrid}}
\usepackage{verbatim}
\usepackage{color}
\usepackage{amsmath,amsfonts,amssymb}
\providecommand{\alert}[1]{\textbf{#1}}

\title{Climate Change Health Impact Assessments:  Farmer Suicide and Drought Case Study.}
\author{Ivan Hanigan$^1$, David Fisher$^2$, Steven McEachern$^3$}
\date{\today}
\hypersetup{
  pdfkeywords={},
  pdfsubject={},
  pdfcreator={Emacs Org-mode version 7.9.3f}}

\institute[NCEPH]{$^1$National Centre for Epidemiology and Population Health (ANU) \\ $^2$Information Technology Services (ANU) \\ $^3$Australian Data Archives (ANU)}
\begin{document}

\maketitle

% Org-mode is exporting headings to 3 levels.



\section{Aim}
\label{sec-1}
\begin{frame}
\frametitle{Aim}
\label{sec-1-1}

\begin{itemize}
\item General tools for Climate Change Health Impact Assessments (CCHIA)
\item Enhanced capacity for experimentation, reviews, revisions and re-iterations
\end{itemize}

Current approach:
\begin{itemize}
\item historical baseline exposure-response functions, control for some covariates
\item use response function with changed exposures and population at risk
\end{itemize}
\end{frame}
\section{Methods}
\label{sec-2}
\begin{frame}
\frametitle{Historical (Hanigan et al, 2012, \emph{PNAS}, 109)}
\label{sec-2-1}

\begin{footnotesize}
\begin{itemize}
\item {\color{red}Restricted Health and Drought data} and 
\item {\color{blue}Less Restricted Population data} 
\end{itemize}
(Colours refer to data storage and access rules shown in Figure 1).
\begin{eqnarray*}
        log({\color{red} O_{ijk}})  & = & s({\color{red}ExposureVariable})  + {\color{blue} OtherExplanators}  \\
        & &   + AgeGroup_{i} + Sex_{j} \\
        & &   + {\color{blue} SpatialZone_{k}}  \\
        & &  + sin(Time \times 2 \times \pi) + cos(Time \times 2 \times \pi) \\
        & &  + Trend \\
        & &   + offset({\color{blue} log(Pop_{ijk})})\\
\end{eqnarray*}
\end{footnotesize}
\begin{tiny}
\noindent Where:\\
        \indent ${\color{red}O_{ijk}}$ = Outcome (counts) by Age$_{i}$, Sex$_{j}$ and SpatialZone$_{k}$ \\
        \indent {\color{red}ExposureVariable} = Data with {\color{red}Restrictive Intellectual Property~(IP)} \\
        \indent {\color{blue}OtherExplanators} = Other {\color{blue}Less Restricted}  Explanatory variables \\
        \indent s( ) = penalized regression splines \\
        \indent ${\color{blue} SpatialZone_{k}}$  = {\color{blue} Less Restricted} data representing the $SpatialZone_{k}$  \\
        \indent Trend = Longterm smooth trend(s) \\
        \indent ${\color{blue}Pop_{ijk}}$ = interpolated Census populations, by time in each group\\
\end{tiny}
\end{frame}
\begin{frame}
\frametitle{Historical (Hanigan et al, 2012, \emph{PNAS}, 109)}
\label{sec-2-2}

\begin{itemize}
\item 38 years suicide rates with drought by 11 regions, age and sex
\item Estimated 9\% in rural males aged 30-49 due to drought over the period
\item Increased for rural males 10-29 y
\item Association with hot temp + spring
\end{itemize}
\end{frame}
\begin{frame}
\frametitle{Future (Bambrick et al, 2008, Garnaut Review)}
\label{sec-2-3}

\begin{footnotesize}
$$Y_{ijk}=\sum_{lm}(e^{(\beta_{ijk} \times {\color{red} X_{lm}})} - 1) \times {\color{red}BaselineRate_{jkl}} \times {\color{blue} Population_{jklm}}$$
\noindent Where:\\
$\beta_{ijk}$ = the ExposureVariable coefficient for zone$_i$, age$_j$ and sex$_{k}$ \\
${\color{red}X_{lm}}$ = Projected Future ExposureVariables {\color{red} with Restrictive IP} \\
{\color{red}BaselineRate$_{jkl}$} = {\color{red}avgDeathsPerTime}/{\color{blue}avgPopPerTime} in age$_j$, sex$_k$ and zone$_l$ \\
{\color{blue}Population$_{jklm}$} = projected populations by age$_j$, sex$_k$, zone$_l$ and time$_m$ {\color{blue} (With Less Restrictions)}\\

\end{footnotesize}
\end{frame}
\section{Results}
\label{sec-3}
\begin{frame}
\frametitle{Drought-suicide response function}
\label{sec-3-1}

\begin{figure}[!h]
\centering
\includegraphics[width=.5\textwidth]{Figure1.png}
\caption{Figure1.png}
\label{fig:Figure1.png}
\end{figure}
\end{frame}
\begin{frame}
\frametitle{Criticism}
\label{sec-3-2}

This model is too static, reductionist, reality is more complex. Need to work more on interactions with non-climate factors especially: 
\begin{itemize}
\item Natural capital
\item Financial capital
\item Social capital
\item Physical capital and
\item Human capital
\end{itemize}
\end{frame}
\section{Conclusion}
\label{sec-4}
\begin{frame}
\frametitle{Conclusion}
\label{sec-4-1}

\begin{itemize}
\item Drought is related to increased suicide risk in Australia
\item Future Drought associated deaths can be calculated
\item These estimates will be very uncertain, contentious and difficult to justify
\item New technology is needed to enable rigorous and transparent exploration
\end{itemize}
\end{frame}
\section{Acknowledgements}
\label{sec-5}
\begin{frame}
\frametitle{Acknowledgements}
\label{sec-5-1}

\includegraphics[width=4cm]{ANU_LOGO_cmyk_56mm.png}
\includegraphics[width=2cm]{andslogo.pdf}
\includegraphics[width=3cm]{deptlogo.pdf} \\
\begin{footnotesize}
This project is supported by the Australian National Data Service through the National Collaborative Research Infrastructure Strategy Program and the Education Investment Fund (EIF) Super Science Initiative.

More information from \texttt{ivan.hanigan@gmail.com} or at \texttt{[[http://opensoftware-restricteddata.github.io][http://opensoftware-restricteddata.github.io]]}

\end{footnotesize}
\end{frame}
\section{References}
\label{sec-6}
\begin{frame}
\frametitle{References}
\label{sec-6-1}

\begin{footnotesize}
\begin{thebibliography}{1}

\bibitem{Peng2011}
Roger~D Peng.
\newblock {Reproducible research in computational science.}
\newblock {\em Science (New York, N.Y.)}, 334(6060):1226--7, December 2011.

\bibitem{Hanigan2012b}
I.~C. Hanigan, C.~D. Butler, P.~N. Kokic, and M.~F. Hutchinson.
\newblock {Suicide and drought in New South Wales, Australia, 1970-2007}.
\newblock {\em Proceedings of the National Academy of Sciences}, pages
  1112965109--, August 2012.

\bibitem{Climate2008}
Hilary~J Bambrick, Keith B~G Dear, RE~Woodruff, Ivan~Charles Hanigan, and
  Anthony~J McMichael.
\newblock {The impacts of climate change on three health outcomes:
  temperature-related mortality and hospitalisations, salmonellosis and other
  bacterial gastroenteritis, and population at risk from dengue.}
\newblock Technical report, Garnaut Climate Change Review, Canberra, 2008.

\end{thebibliography}
\end{footnotesize}
\end{frame}

\end{document}
